% Einleitung und evtl. Erklärung von zusätzlichen Themen die für das Verständnis des Design-Patterns sinnvoll, nützlich oder notwendig sind
\chapter{Einleitung}\label{ref:einleitung}
Um die, bereits in der \hyperref[ref:motivation]{Motivation} erwähnten, vielen Abhängigkeiten zwischen den einzelnen Klassen aufzulösen und somit \gls{loose_coupling_g} herzustellen, wird ein Mediator eingesetzt. Wie auch in der Grafik \ref{fig:noMediatorVsMediator} veranschaulicht, interagieren Objekte dann nicht mehr direkt miteinander, sondern immer über den Mediator. Dies führt sowohl zu einigen Vor- als aber auch Nachteilen (siehe: \hyperref[ref:vorteileNachteile]{Vor- und Nachteile vom Mediator}). Weiters gehört der Mediator zur Gruppe der \gls{behavioralpattern_g}.
\cfigure{0.95}{chapters/chapter2_introduction/assets/NoMediatorVsMediator}{Vergleich kein Mediator / Mediator - Extremfall}{noMediatorVsMediator}
\autocite[vgl.][S. 273ff]{gangoffour_book}
%
\section{zusätzlichen Themen}
Um das Mediator Design-Pattern besser verstehen zu können, benötigt es auch das Verständnis folgender Konzepte:
\subsection{lose Kopplung}\label{ref:looseCoupling}
Bei Kopplung im Software-Bereich unterscheidet man meist zwischen starker und loser Kopplung. Als Kopplung wird der Grad der Ab- /bzw. Unabhängigkeit zwischen zwei Komponenten beschrieben. Ziel ist es, möglichst unabhängige Software-Komponenten und somit modulare Software zu erstellen. Die Änderung einer Komponente sollte möglichst keine Auswirkung auf andere haben. Dies hat unter anderem folgende Vorteile: 
\begin{outline}\label{ref:prosLoseKopplung}
    \1 bessere Wartbarkeit: nicht jede kleine Änderung benötigt Änderung am \enquote{gesammten} System. z.B. Datenquelle kann einfach an einer \textbf{einzigen} Stelle ausgetauscht werden und alle funktioniert weiterhin
    \1 Wiederverwendbarkeit: einzelne Komponenten können in neuen Projekten verwendet werden ohne das \enquote{komplette} alte Programm kopieren zu müssen
    \1 Testbarkeit: Während Unit-Test können einzelne Komponenten (oft z.B. Datenbank) gegen emulierte Varianten ausgetauscht werden.
    \1 ...
\end{outline}

\medskip
\noindent\autocite[vgl.][]{loosecoupling_reconceptualization}, basierend auf \autocite[vgl.][S. 101ff]{meilir_loosecoupling}
\medskip

\textbf{Lose Kopplung in Java}: \\
Lose Kopplung in Java (und auch vielen anderen Sprachen) wird oft über Interfaces (bzw. in anderen Sprachen dazu äquivalentes) realisiert. Hier ein simples Beispiel eines Taschenrechners:
\cinputminted{java}{chapters/chapter2_introduction/include/tight.java}{starke Kopplung Beispiel}{tight}
Änderungen in der Calculator-Klassen (ändern der Methoden-Namen / Ändern der Parameter: z.B. Änderung auf Array, damit beliebige Anzahl übergeben werden kann), benötigen in diesem simplen Beispiel zumindest Änderung der CalcularorApp-Klasse oder im schlimmsten Fall sogar bis zur Main-Methode, wenn die CalculatorApp-Klasse nicht als Facade agiert und dann z.B. ebenfalls ein Array erwartet.

Eine bessere Lösung dieser Problemstellung könnte wie folgt aussehen:
\cinputminted{java}{chapters/chapter2_introduction/include/loose.java}{lose Kopplung Beispiel}{loose}

Jedes \enquote{new} sorgt wieder für einen gewissen Grad an starker Kopplung, daher wird das Konzept der Interfaces dann noch mit Dependency-Injection kombiniert, um einen maximal möglichen Grad an Unabhängigkeit zu gewährleisten.
\newpage
% 
\section{Verwendungsbeispiele aus der Praxis}\label{ref:VerwendungPraxis}
Das Mediator Pattern ist in der Praxis häufig im Einsatz. Unter anderem findet es bei GUI-Anwendungen (auch im JDK) sehr gern, aber natürlich auch bei anderen Problemstellungen Anwendung.
\begin{outline}
    \1 MVC: in fast allen Fällen agiert der Controller beim MVC Pattern als Mediator zwischen Model und View. z.B. in einer Swing-Applikation (oder auch Web-Applikation wie z.B. Angular) kümmert er sich um Data-Binding mit dem View, aktiviert / deaktiviert z.B. einen \enquote{absenden} Button basierend darauf, ob bereits alle notwendigen Eingaben getätigt wurden und sendet dann beim Klick dieses die Daten weiter z.B. Persistierung der Daten in Datenbank
    (bei \enquote{HTML-Generatoren}, wie z.B. JSF wird die Backing-Bean teils als Mediator gesehen / teils nicht, da sie oft \enquote{nur} Daten für den JSF->Html Generator bereitstellt)
    \1 Swing: Event Dispatch Thread: 1 Thread, der sich um sämtliche GUI-Relevante dinge, kümmert - alles wird NUR von diesem Thread ausgeführt: Änderungen / Benutzer-Events / ... → sicheres mutlithreading (wenn mehrere Threads am UI Änderungen durchführen, führt die meist nur zu Problemen)
    \1 AWT ActionListener: auch hier wird intern ein Mediator verwendet, der bei Events im GUI benachrichtigt wird und dann entsprechend das Event an die einzelnen ActionListener weiterleitet
    \1 java.util.concurrent: Mediator koordiniert Kommunikation zwischen Threads. Synchronisation durch z.B. Blocking-Queues, ...
    \1 Javascript event loop: Der Javascript Event Loop kann ebenfalls als Mediator gesehen werden. Er erhält von Objekten Ereignisse / Aufgaben wie z.B. senden einer asynchronen HTTP-Request, und kümmert sich darum, sobald das Ergebnis des Servers da ist, eine entsprechende Callback-Funktion aufzurufen
    \1 ...
\end{outline}
\subsection{konkreten Anwendung: Smart-Home-Hub}
In Grafik \ref{fig:smartHomeHub} ist eine einfache konkrete Anwendung des Mediator Design Patterns zu sehen: Ein zentraler Smart-Home-Hub wie z.B. Home-Assistant / ..., der als Mediator zwischen den einzelnen Komponenten dient.
\cfigure{0.8}{chapters/chapter2_introduction/assets/MediatorSmartHome}{grafische Darstellugn Smart-Home-Hub-Mediator}{smartHomeHub}
Beispielsweise könnte hier folgende Kommunikation ablaufen:
\begin{outline}
    \1 \textbf{Smartphone App}: interagiert mit Zentrale:
        \2 steuert Komponenten: Licht, Heizung, ...
        \2 erhält Informationen: Information über aktuelle Temperatur → Push-Benachrichtigung, wenn zu kalt, ...
    \1 \textbf{smarter Heizkörperregler}: wird vom Hub gesteuert und stellt die entsprechend gewünschte Temperatur ein
    \1 \textbf{smarte Lampe}: wird vom Hub gesteuert, wenn Person auf Lichtschalter drückt aber auch z.B. automatisch durch Bewegungsmelder, automatische Aufwachroutine, ...
    \1 \textbf{smarter Lichtschalter} (hier z. B.: Sonoff-NSPanel): lässt sowohl Licht als auch andere Dinge steuern / Informationen anzeigen
\end{outline}
% 
\section{Vor- und Nachteile vom Mediator}\label{ref:vorteileNachteile}
\filbreak
\subsection{Vorteile}
\begin{outline}
    \1 Lose Kopplung: Da der Mediator \gls{loose_coupling_g} fördert, sind die Vorteile dieses (\hyperref[ref:prosLoseKopplung]{Vorteile llose Kopplung}) auch auf den Mediator anwendbar.
        \2 Wartbarkeit: Änderungen in einer Colleague-Klasse benötigt nur Änderung im Mediator. Andere Colleague-Klassen bedürfen keinen Änderungen
        \2 Wiederverwendbarkeit: Colleague und Mediator Klassen können unabhängig wiederverwendet / ausgetauscht werden
        \2 Testbarkeit: Es kann einfach eine andere Implementierung des Mediator-Interfaces verwendet werden.
    \1 Änderbarkeit / Erweiterbarkeit: Anderes Verhalten bzw. weitere Funktionen benötigt nur Änderung / erstellen einer Sub-Klasse bzw. Implementierung des Interfaces des Mediators. Colleague-Klassen können oft 1 zu 1 wiederverwendet werden,
    \1 zentrale Stelle zur Verwaltung der Interaktions-Logik zwischen den Objekten. Auch sind dadurch die Komplexität der Colleague Objekte selbst,
        \2 bessere Übersicht / Lesbarkeit / Verständnis,
        \2 bessere Wartbarkeit,
        \2 vereinfachte Interaktionslogik: vorher Many-To-Many Beziehungen zwischen Colleagues - jetzt nur mehr One-To-One/Many: Colleague -(one)> Mediator -(many)> anderer Colleague
    \1 ...
\end{outline}

\filbreak
\subsection{Nachteile}
\begin{outline}
    \1 Auslagerung der Komplexität der Interaktion zwischen Objekten in Mediator lässt diese nicht magisch verschwinden. Oft wird der Mediator selbst dann sehr komplex → wird zum Monolith, ohne den nichts geht → wieder nicht gut für Wiederverwendbarkeit
    \1 \enquote{god object}: im Worstcase wird der Mediator zu einem oft als \enquote{God Object} bezeichneten Objekt. Also einem Objekt, dass alles kann / macht. Eine Klasse mit \textbf{allen (tausenden) Funktionen} der Applikation.
    \1 geringere Flexibilität: Interaktion muss / sollte immer über Mediator passieren
    \1 Mediator verletzt 1. SOLID Design Prinzip (Single-responsibility): gibt meist deutlich mehr als einen Grund den Mediator zu ändern / hat mehr als 1 Aufgabe
    \1 Skalierbarkeit: Mediator kann zum Bottleneck werden
    \1 einige kleinere (\enquote{unwichtige}) Nachteile:
        \2 KISS-Prinzip / Overengineering: für kleine Applikationen erhöht der Mediator (wie auch andere Design-Patterns) nur Sinnlos den Aufwand der Entwicklung. Es bedarf hier generell immer einer Abwägung: Ja, Design-Patterns sind wichtig und sinnvoll, aber man kann es auch übertreiben
        \2 (zusätzliches Objekt -> \enquote{höherer} Ressourcenbedarf)
    \1 ...
\end{outline}

\autocite[vgl.][S. 273ff]{gangoffour_book} + eigene Ergänzungen
% 
\section{Verwandte Design-Patterns}
Der Mediator ähnelt dem Facade-Pattern. Allerdings unterscheidet er sich wie folgt: Einen Mediator kann man sich ein bisschen wie eine BI-direktionale-Fassade vorstellen. Eine Facade stellt z.B. eine einfachere Schnittstelle für eine komplexe Library bereit. Ein Mediator vereinfacht die Kommunikation \textbf{zwischen} (also in beide Richtungen!) zwei Objekten. Zusätzlich kann ein Mediator zusätzliche Funktionalität bereitstellen, die von den einzelnen Colleague Objekten selbst nicht bereitgestellt wird / bereitgestellt werden kann. 
\autocite[vgl.][S. 273ff]{gangoffour_book}