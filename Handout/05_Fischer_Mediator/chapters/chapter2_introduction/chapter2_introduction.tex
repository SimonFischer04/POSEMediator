% Einleitung und evtl. Erklärung von zusätzlichen Themen die für das Verständnis des Design-Patterns sinnvoll, nützlich oder notwendig sind

% Behavioral ...
\chapter{Einleitung}
Um die, bereits in der \hyperref[ref:motivation]{Motivation} erwähnten, vielen Abhängigkeiten zwischen den einzelnen Klassen aufzulösen und somit \gls{loose_coupling_g} herzustellen, wird ein Mediator eingesetzt. Wie auch in der Grafik \ref{fig:noMediatorVsMediator} veranschaulicht, interagieren Objekte dann nicht mehr direkt miteinander, sondern immer über den Mediator. Dies führt sowohl zu einigen Vor- als aber auch Nachteilen (siehe: \hyperref[ref:vorteileNachteile]{Vor- und Nachteile vom Mediator}). Weiters gehört der Mediator zur Gruppe der \gls{behavioralpattern_g}.
\cfigure{0.95}{chapters/chapter2_introduction/assets/NoMediatorVsMediator}{Vergleich kein Mediator / Mediator - Extremfall}{noMediatorVsMediator}
\autocite[vgl.][S. 273ff]{gangoffour_book}
%
\section{zusätzlichen Themen}
Um das Mediator Design-Pattern besser verstehen zu können, benötigt es auch das Verständnis folgender Konzepte:
\subsection{lose Kopplung}\label{ref:looseCoupling}
Bei Kopplung im Software-Bereich unterscheidet man meist zwischen starker und loser Kopplung. Als Kopplung wird der Grad der Ab- /bzw. Unabhängigkeit zwischen zwei Komponenten beschrieben. Ziel ist es, möglichst unabhängige Software-Komponenten und somit modulare Software zu erstellen. Die Änderung einer Komponente sollte möglichst keine Auswirkung auf andere haben. Dies hat unter anderem folgende Vorteile: 
\begin{outline}\label{ref:prosLoseKopplung}
    \1 bessere Wartbarkeit: nicht jede kleine Änderung benötigt Änderung am \enquote{gesammten} System. z.B. Datenquelle kann einfach an einer \textbf{einzigen} Stelle ausgetauscht werden und alle funktioniert weiterhin
    \1 Wiederverwendbarkeit: einzelne Komponenten können in neuen Projekten verwendet werden ohne das \enquote{komplette} alte Programm kopieren zu müssen
    \1 Testbarkeit: Während Unit-Test können einzelne Komponenten (oft z.B. Datenbank) gegen emulierte Varianten ausgetauscht werden.
    \1 ...
\end{outline}

\medskip
\noindent\autocite[vgl.][]{loosecoupling_reconceptualization}, basierend auf \autocite[vgl.][S. 101ff]{meilir_loosecoupling}
\medskip

\textbf{Lose Kopplung in Java}: \\
Lose Kopplung in Java (und auch vielen anderen Sprachen) wird oft über Interfaces (bzw. in anderen Sprachen dazu äquivalentes) realisiert. Hier ein simples Beispiel eines Taschenrechners:
\cinputminted{java}{chapters/chapter2_introduction/include/tight.java}{starke Kopplung Beispiel}{tight}
Änderungen in der Calculator-Klassen (ändern der Methoden-Namen / Ändern der Parameter: z.B. Änderung auf Array, damit beliebige Anzahl übergeben werden kann), benötigen in diesem simplen Beispiel zumindest Änderung der CalcularorApp-Klasse oder im schlimmsten Fall sogar bis zur Main-Methode, wenn die CalculatorApp-Klasse nicht als Facade agiert und dann z.B. ebenfalls ein Array erwartet.

Eine bessere Lösung dieser Problemstellung könnte wie folgt aussehen:
\cinputminted{java}{chapters/chapter2_introduction/include/loose.java}{lose Kopplung Beispiel}{loose}

Jedes \enquote{new} sorgt wieder für einen gewissen Grad an starker Kopplung, daher wird das Konzept der Interfaces dann noch mit Dependency-Injection kombiniert, um einen maximal möglichen Grad an Unabhängigkeit zu gewährleisten.
% 
\section{Verwendungsbeispiele aus der Praxis}
\todo{write}
% 
\section{Vor- und Nachteile vom Mediator}\label{ref:vorteileNachteile}
Zuerst: da der Mediator \gls{loose_coupling_g} fördert, sind die Vorteile dieses (\hyperref[ref:prosLoseKopplung]{Vorteile llose Kopplung}) auch auf den Mediator anwendbar.

\paragraph{Vorteile}

\todo{god object...}
% 
\section{Verwandte Design-Patterns}
Der Mediator ähnelt der Facade. Allerdings unterscheidet er sich wie folgt:
\todo{write this}